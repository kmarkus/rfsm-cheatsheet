\documentclass[10pt,landscape]{article}
 \usepackage{multicol}
\usepackage{calc}
\usepackage{ifthen}
\usepackage[landscape]{geometry}
\usepackage{amsmath,amsthm,amsfonts,amssymb}
\usepackage{color,graphicx,overpic}
\usepackage{hyperref}
\usepackage{fancyvrb}
\usepackage{url}

\include{pygmentize}

\pdfinfo{
  /Title (rfsm-cheatsheat.pdf)
  /Creator (TeX)
  /Producer (pdfTeX 1.40.0)
  /Author (Markus Klotzbuecher)
  /Subject (rFSM cheatsheet)
  /Keywords (pdflatex, latex,pdftex,tex)}

% This sets page margins to .5 inch if using letter paper, and to 1cm
% if using A4 paper. (This probably isn't strictly necessary.)
% If using another size paper, use default 1cm margins.
\ifthenelse{\lengthtest { \paperwidth = 11in}}
    { \geometry{top=.5in,left=.5in,right=.5in,bottom=.5in} }
    {\ifthenelse{ \lengthtest{ \paperwidth = 297mm}}
        {\geometry{top=1cm,left=1cm,right=1cm,bottom=1cm} }
        {\geometry{top=1cm,left=1cm,right=1cm,bottom=1cm} }
    }

% Turn off header and footer
\pagestyle{empty}

% Redefine section commands to use less space
\makeatletter
\renewcommand{\section}{\@startsection{section}{1}{0mm}%
                                {-1ex plus -.5ex minus -.2ex}%
                                {0.5ex plus .2ex}%x
                                {\normalfont\large\bfseries}}
\renewcommand{\subsection}{\@startsection{subsection}{2}{0mm}%
                                {-1explus -.5ex minus -.2ex}%
                                {0.5ex plus .2ex}%
                                {\normalfont\normalsize\bfseries}}
\renewcommand{\subsubsection}{\@startsection{subsubsection}{3}{0mm}%
                                {-1ex plus -.5ex minus -.2ex}%
                                {1ex plus .2ex}%
                                {\normalfont\small\bfseries}}
\makeatother

% Define BibTeX command
\def\BibTeX{{\rm B\kern-.05em{\sc i\kern-.025em b}\kern-.08em
    T\kern-.1667em\lower.7ex\hbox{E}\kern-.125emX}}

% Don't print section numbers
\setcounter{secnumdepth}{0}


\setlength{\parindent}{0pt}
\setlength{\parskip}{0pt plus 0.5ex}

%My Environments
\newtheorem{example}[section]{Example}
% -----------------------------------------------------------------------

\begin{document}
\raggedright
\footnotesize
\begin{multicols}{3}


% multicol parameters
% These lengths are set only within the two main columns
%\setlength{\columnseprule}{0.25pt}
\setlength{\premulticols}{1pt}
\setlength{\postmulticols}{1pt}
\setlength{\multicolsep}{1pt}
\setlength{\columnsep}{2pt}

\begin{center}
     \Large{\underline{rFSM Cheatsheet}}\\
     \small{(beta5)} \\
\end{center}

\section{Specifying rFSM Models}

\textbf{Synopsis:}\\
\begin{verbatim}
    return rfsm.state {
       ...
    }
\end{verbatim}

\textbf{Decription:} 

A rFSM model is contained in a file that returns the toplevel model in
a state.

\subsection{Toplevel FSM Configuration}

The fields may defined at the root of an fsm:␘

\begin{itemize}
\item \texttt{\texttt{getevents()}}: function that will be executed
  during step/run and must return a table of events (index part only).
\item \texttt{dbg}, \texttt{info}, \texttt{err}: log functions. Must
  takes multiple arguments and log them.
\end{itemize}


%%% States
\subsection{States}

\textbf{Synopsis:}\\
\begin{verbatim}
    rfsm.state{
       entry=function() do_this() end
       exit=function() do_that() end,
    
       substate1=rfsm.state{...}, 
       ...
       transition{...}
       ...
    }
\end{verbatim}


\textbf{Fields:}\\
\begin{itemize}
\item \texttt{entry(fsm, state, 'entry'}: entry function 
\item \texttt{exit(fsm, state, 'exit'}: entry function 
\item \texttt{doo(fsm, state, 'doo'}: doo function/coroutine (only permited for leaf states)
\end{itemize}


\textbf{Description:}\\
The \texttt{state} is the basic building block. If a state has
substates it is called a \textbf{composite state}. Otherwise it is
called a \textbf{leaf state}.\\ 

\subsection{Transitions}

\textbf{Synopsis:}\\

\begin{verbatim}
    rfsm.transition{ src=''stateX'',
                     tgt=''stateY'',
                     events={...}},
    rfsm.trans{ ... } -- short form
\end{verbatim}


\textbf{Fields}:\\
\begin{itemize}
\item \texttt{src=<state-ref>}:source state reference
\item \texttt{tgt=<state-ref>}: target state reference
\item \texttt{events=\{event1, event2, ...\} }: list of triggering events
\item \texttt{guard }: guard function. Stateless function that returns \texttt{true} or \texttt{false}.
\item \texttt{pn=<number>}: Priority number. Transitions with larger
  numbers take priority in case of conflicting transitions.
\end{itemize}

\subsection{Connectors}
\label{sec:connectors}

\textbf{Synopsis:}\\
\begin{verbatim}
    rfsm.connector{}
    rfsm.conn{} -- short form
\end{verbatim}

\textbf{Fields}: none.

\textbf{Description:}\\

Connectors can be used to build composite transitions by chaining
multiple connectors, or for defining different entry and exit points
of composite states. 


\section{Loading and Instatiating rFSM Models}
\label{sec:running-rfsm}

\begin{itemize}
\item \texttt{model=rfsm.load(filename)} Load a rFSM Model from a file
\item \texttt{rfsm.init(model)} Instatiate and validate a rFSM model
\end{itemize}


\section{Running rFSM instances}
\label{sec:runn-rfsm-inst}

\begin{itemize}
\item \texttt{rfsm.step(fsm)}: \texttt{step} a rFSM instance. This
  will execute at most one transition or one doo cycle.

\item \texttt{rfsm.step(fsm, N)}: \texttt{step} as above, but will
  perform at most \texttt{N} transitions or doo cycles.

\item \texttt{rfsm.run(fsm)} will run step until there are no new
  events and no active \texttt{doo} function.

\item \texttt{rfsm.send\_events(fsm, event1, event2, \dots)}: send events
  to the internal fsm queue. Mostly used for debugging. Normally the
  \texttt{getevents} hook shall be preferred.
\end{itemize}

\section{Miscellaneous}
\label{sec:miscellaneous}

\subsection{Tracing}
\label{sec:debugging}

Enable state entry and exit debug messages:

\begin{verbatim}
require "rfsmpp"
fsm.dbg=rfsmpp.gen_dbgcolor("fsmX", 
                            { STATE_ENTER=true,
                              STATE_EXIT=true }, false)
\end{verbatim}

\subsection{Example}
\label{sec:example-1}

\begin{verbatim}
local state, trans, conn = rfsm.state, rfsm.conn, rfsm.trans

return rfsm.state {
   dbg=rfsmpp.gen_dbgcolor("sample-fsm",
                            { STATE_ENTER=true,
                              STATE_EXIT=true }, false),
   entry=function() enable_robot() end,
   entry=function() disable_robot() end,

   stopped = state {},

   moving = state {
      entry=function() start_motor() end,
      exit=function() stop_motor() end,
      
      doo = function()
        while true do
           compute_next_pos()
           rfsm.yield(true)
        end,
   }

   trans{ src="initial", tgt="stopped" },
   trans{ src="stopped", tgt="moving", events={"e_start"} },
   trans{ src="moving", tgt="stopped", events={"e_stop"} },
}
\end{verbatim}

\subsection{Links}
\label{sec:links}

\url{www.orocos.org/rfsm}
\url{http://www.orocos.org/wiki/orocos/toolchain/LuaCookbook}

% You can even have references
\rule{0.3\linewidth}{0.25pt}
\scriptsize
\bibliographystyle{abstract}
\bibliography{refFile}
\end{multicols}
\end{document}

%%% Local Variables: 
%%% mode: latex
%%% TeX-master: t
%%% End: 
